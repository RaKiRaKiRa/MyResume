% !TEX TS-program = xelatex
% !TEX encoding = UTF-8 Unicode
% !Mode:: "TeX:UTF-8"

\documentclass{resume}
\usepackage{zh_CN-Adobefonts_external} % Simplified Chinese Support using external fonts (./fonts/zh_CN-Adobe/)
% \usepackage{NotoSansSC_external}
% \usepackage{NotoSerifCJKsc_external}
% \usepackage{zh_CN-Adobefonts_internal} % Simplified Chinese Support using system fonts
\usepackage{linespacing_fix} % disable extra space before next section
\usepackage{cite}

\begin{document}
\pagenumbering{gobble} % suppress displaying page number

\name{王一琛}

\basicInfo{
  \email{2018216138@tju.edu.cn} \textperiodcentered\ 
  \phone{(+86) 132-1866-2981} \textperiodcentered\ 
 }
 
\section{\faGraduationCap\  教育背景}
\datedsubsection{\textbf{天津大学}}{2018.9 -- 至今 \,\,\,}
\textit{在读硕士研究生}\ 计算机技术, 预计 2021年毕业
\datedsubsection{\textbf{河海大学}}{2014.9 -- 2018.6}
\textit{学士}\ 通信工程

\section{\faUsers\ 实习经历}
\datedsubsection{\textbf{字节跳动}}{2020年3月 -- 至今}
\role{实习}{字节跳动-互娱研发-数据策略部门-后端开发工程师}
\begin{onehalfspacing}
\begin{itemize}
  \item 参与飞盘平台功能建设
  \begin{itemize}
    \item 参与平台迁移,将所有数据迁移到新平台,并实现数据管理后端接口
    \item 完善thrift\_invoker(自动生成client、编译、重启的服务),包括支持新平台数据、更新kitool版本、支持使用相同IDL文件的PSM调用
    \item 使飞盘平台策略脚本支持函数继承,简化脚本复杂度
    \item 将与CRUD相关接口提取到OpenAPI,并通过标签进行鉴权,上游仅可以操作包含上游对应标签的工具
  \end{itemize}
  \item 参与飞盘平台安全建设
  \begin{itemize}
    \item 完善thrift\_invoker,包括支持支持回滚、安全升级、初次启动时不注册服务发现
    \item 为飞盘平台策略提供单元测试,保证测试的完备性。
    \item 实现对敏感资源的权限验证
  \end{itemize}
\end{itemize}
\end{onehalfspacing}

\section{\faCogs\ 个人项目}
\datedsubsection{\textbf{基于Reactor的多线程网络库 Cyclone} }{2019年06月 -- 2019年12月}
\role{C++, Linux}{https://github.com/RaKiRaKiRa/Cyclone} 
\begin{onehalfspacing}
\begin{itemize}
  \item 实现对互斥锁、条件变量、线程、线程池、任务队列、单例模式、守护进程、应用层缓存、异步日志等常用方法的封装
  \item 封装Linux下的timerfd,使用std::set实现定时器,并进一步封装成时间轮用于关闭超时请求
  \item 使用epoll(LT)+非阻塞IO实现Reactor模式,结合one-loop-per-thread模型实现服务器框架,用Round-Robin算法分配连接实现负载均衡
  \item 使用任务队列解决跨线程调用,并通过eventfd实现线程异步唤醒
  \item 基于网络库实现静态http服务器,并基于该服务器搭建博客( http://blog.RaKiRaKiRa.top )
  \item 基于以JSON-RPC 2.0协议的序列化/反序列化方案实现简易RPC库(github.com/RaKiRaKiRa/Cycl\\noneRPC)
\end{itemize}
\end{onehalfspacing}

\section{\faInfo\ IT技能}
% increase linespacing [parsep=0.5ex]
\begin{itemize}[parsep=0.5ex]
  \item 熟练使用C++,掌握Golang,熟悉常用数据结构和算法
  \item 熟悉计算机网络及Socket编程
  \item 熟悉操作系统,熟悉多线程编程
  \item 了解Linux内核,了解Redis,了解MySQL
  %\item 了解MySQL的常用语句编写、索引和优化
\end{itemize}



%\section{\faHeartO\ 获奖情况}
%\datedline{\textit{第一名}, xxx 比赛}{2013 年6 月}
%\datedline{其他奖项}{2015}

% \section{\faInfo\ 其他}
% % increase linespacing [parsep=0.5ex]
% \begin{itemize}[parsep=0.5ex]
%   \item 技术博客: http://blog.RaKiRaKiRa.top
%   \item GitHub:  \,\,\, \,https://github.com/RaKiRaKiRa
% \end{itemize}

%% Reference
%\newpage
%\bibliographystyle{IEEETran}
%\bibliography{mycite}
\end{document}
